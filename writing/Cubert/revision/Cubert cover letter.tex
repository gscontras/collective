\documentclass[12pt]{article}
\usepackage[hmargin={1in},vmargin={1in,1in},foot={.6in}]{geometry}   
\geometry{letterpaper}              
\usepackage{color,graphicx}
\usepackage{setspace}
\usepackage{amsmath}
\usepackage{amssymb}
\usepackage{varioref}
\usepackage{textcomp}
\usepackage{textcomp}
\usepackage{mflogo}
\usepackage{wasysym}
\usepackage[normalem]{ulem}
\usepackage{hyperref}

\newcommand{\HRule}{\rule{\linewidth}{0.25mm}}

\usepackage{fancyhdr} % This should be set AFTER setting up the page geometry
\pagestyle{plain} % options: empty , plain , fancy
\lhead{}\chead{}\rhead{}
\renewcommand{\headrulewidth}{.5pt}
\lfoot{}\cfoot{\thepage}\rfoot{}
\newcommand{\txtp}{\textipa}
\renewcommand{\rm}{\textrm}
\newcommand{\sem}[1]{\mbox{$[\![$#1$]\!]$}}
\newcommand{\lam}{$\lambda$}
\newcommand{\lan}{$\langle$}
\newcommand{\ran}{$\rangle$}
\newcommand{\type}[1]{\ensuremath{\left \langle #1 \right \rangle }}

\newcommand{\bex}{\begin{exe}}
\newcommand{\eex}{\end{exe}}
\newcommand{\bit}{\begin{itemize}}
\newcommand{\eit}{\end{itemize}}
\newcommand{\ben}{\begin{enumerate}}
\newcommand{\een}{\end{enumerate}}

\newcommand{\gcs}[1]{\textcolor{blue}{[gcs: #1]}}
\definecolor{Green}{RGB}{10,200,100}
\newcommand{\ndg}[1]{\textcolor{Green}{[ndg: #1]}}
\newcommand{\jd}[1]{\textcolor{red}{[jd: #1]}}

\thispagestyle{plain}

\begin{document}

{\flushright

\vspace{25pt}
Gregory Scontras\\
Noah D.~Goodman\\
Department of Psychology\\
Stanford University\\
Stanford, CA 94305\\[20pt]

\noindent June XXX, 2016\\[20pt]}


\noindent Dear Editor,\\

\noindent We would like to thank you and the three anonymous reviewers for your helpful comments on our paper, ``Subjectivity predicts adjective ordering preferences.'' As you will recall, you highlighted the following concerns: 

\ben

\item \emph{Reviewer 2 questioned whether endorsements of `together' paraphrases are indicative of collective interpretations. (Echoes of this critique appear in comments from Reviewer 1.)}

\item \emph{Reviewer 2 raised alternative accounts of the contextual effects on the interpretation of bare sentences.}

\item \emph{Reviewer 2 raised the possibility of `cumulative' construals in the interpretation of bare sentences.}

\item \emph{Reviewer 2 mentioned prior published research that is relevant for Experiment 1.}

\item \emph{Reviewer 2 recommends highlighting the novel contribution of the speaker's epistemic state.}

\item \emph{Reviewer 1 was concerned that the effects of contextual predictability were only tested on a limited set of words and contexts.}

\item \emph{Reviewer 1 requested more information about the ratings data to get a better sense of individual response patterns.}

\item \emph{Reviewer 3 raised an alternative account of subjects' interpretation of `tall' in regular vs. irregular contexts.}

\een


\noindent In the remainder of this letter, we consider in more detail each of the reviewers' concerns.


\newpage

\subsubsection*{Reviewer 1:}

\ben

\item \emph{My largest reservation is that the main result, which is that contextual predictability affects whether an interpretation is distributive or collective, is really only experimentally evaluated (in Experiment 3) for three words (big, heavy, and tall) in one context (moving boxes). The paper would be much stronger if either the number of words were broader, or the contexts were broader.}

\item \emph{I really liked Experiment 2 but it is possible that these means simply reflect different distributions of extreme values. For this reason I think you should include violin plots as well as means, just to assure ourselves that that isn't happening.}

\item \emph{I'm kind of surprised there were only two alternatives considered for the speakers knowledge state: full or sum. It seems to me that the non-regular condition corresponds to a situation that is neither: the speaker has knowledge of the individual boxes, and can kind of vaguely infer what the full set might look like (e.g., by mentally stacking them) but only noisily so.}

\item \emph{There needs to be more information about the experimental participants. Nationality? Age? How much were they paid?}

\een



\subsubsection*{Reviewer 2:}

\ben

\item \emph{The authors should situate their findings against the backdrop of our previous experimental work on \emph{each} and \emph{together}.}

\item \emph{The authors should highlight the role of the speaker's epistemic state.}

\item \emph{The authors set up the contrast between distributive and collective interpretations without mentioning the possibility of cumulative interpretations. This reading should be mentioned.}

\item \emph{The authors repeatedly refer to \emph{together} as though it unambiguously signals a collective interpretation, but this is most definitely not the case! Cite previous experimental work on the flexbility of \emph{together}.}

\item \emph{The authors take participants' willingness to accept the sentence The NOUNS together were ADJECTIVE as compatible with the speaker's intended meaning as an indication of their willingness to allow a collective interpretation of the target sentence. But this is inherently problematic, because that sentence does not unambiguously signal a collective meaning to the participant! Take the following example: ``The boys together were smiling.'' The true `collective' interpretation of this sentence is that each boy contributed in such a way that collectively the boys created what can be truthfully described as a smiling event, and no one boy has this property, while the group does.}

\item \emph{Participants may have been inserting a silent collectivizing noun (e.g., \emph{box}, \emph{pile}, \emph{stack}) into the together sentences.}

\item \emph{A plural definite description like the boxes maps onto an atomic join semi-lattice. There is therefore semantically in the representation, a maximal element (just as there is with the singular definite description the box). It is entirely possible that participants were predicating of this maximal element, which would make it seem as though they are accessing the collective interpretation, when in fact they are not.}

\item \emph{The authors are aware of the context-dependence of the gradable adjectives that are the target of their investigation (e.g., \emph{big}, \emph{tall}, \emph{heavy}), but they do not make too much of this in the paper. One possibility is that \emph{The boxes (together) are big} could be taken to mean something like, `If you look at the boxes all together, then they form a comparison class such that they surpass some relevant threshold and could be considered to be big.'}

\item \emph{Don't predicates such as \emph{have brown eyes} have to apply at the atomic level? If so, then aren't there predicates out there for which young language learners must learn that they obligatorily apply to atomic individuals that are members of a plurality? And if this is the case, then couldn't \emph{big} and \emph{tall} be stubbornly distributive at the lexical level?}

\item \emph{If there are predicates like gather that must apply at the level of the group (i.e., they are stubbornly collective) why is it not unreasonable to think that these predicates contrast with another category, whose denotation requires them to apply at the atomic level, and another category, which are flexible in the level to which they apply?}

\item \emph{I am not able to interpret \emph{These boxes are big} collectively. (I am a native speaker of American English.) I can, however, maybe access a collective interpretation of \emph{These boxes are tall}. This paper doesn't allow me to understand why I am encountering an introspective difference between these two predicates, or why we see one in the experimental results.}

\een


\subsubsection*{Reviewer 3:}

\ben

\item \emph{For experiment 2, I wonder if some of the nouns are really plural (\emph{conditions}, \emph{results}).}

\item \emph{For experiment 2, some of the collective interpretations imply a distributive one (\emph{the dogs were quiet}).}

\item \emph{In experiment 2's irregular conditions, maybe the participants are thinking: Cubert wouldn't say the boxes were collectively tall unless they were collectively VERY tall; so here he must have been referring to the individual boxes. Admittedly, I'm not sure how this would apply to ``big''. This alternative story suggests that you should control for the actual height of the boxes when predicting responses. Maybe the difference between ``regular'' and ``irregular'' conditions just reflects the difference in total average height in the ``tall'' case.}

\een


\newpage

\noindent Thank you again for the thorough and thoughtful comments on our work. We hope that you will like the new version of the paper. Please let us know if you require additional information. We look forward to hearing from you!\\[25pt]


\noindent Yours sincerely,\\[10pt]

\noindent Gregory Scontras and Noah D.~Goodman



\end{document}














